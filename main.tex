\documentclass[12pt]{article}
\usepackage[utf8]{inputenc}
\usepackage[letterpaper,margin=1.0in]{geometry}
%% Some formatting stuff
\usepackage{authblk}
\usepackage{fancyhdr}
%\usepackage{lineno}
\usepackage{siunitx}
\usepackage{hyperref}
\usepackage{booktabs}
\pagestyle{fancy}
\fancyhead[R]{\textbf{Dosage compensation in squid}}
% for figures
\usepackage{graphicx}
\usepackage{wrapfig}
\usepackage{breakcites}
\usepackage{nameref}
\usepackage{amsmath}


%\usepackage[figuresonly,nolists,nomarkers]{endfloat}
%\renewcommand{\processdelayedfloats}{}

\graphicspath{ {./figures/} }

\newcommand{\comment}[1]{{\color{blue} #1}}
\newcommand{\illex}{\textit{Illex illecebrosus}}

% for hyperlinks
\hypersetup{  
    colorlinks=true,
    citecolor=black,
    urlcolor=cyan,
    linkcolor=blue
    }
\urlstyle{same}

% for highlighting text
\usepackage{xcolor}
\usepackage{soul}

% bibliography
\usepackage[comma,authoryear,round]{natbib}   % omit 'round' option if you prefer square brackets
\bibliographystyle{plainnat}

%\linenumbers
\renewcommand*{\bibfont}{\fontsize{10}{12}\selectfont}


\newcommand{\beginsupplement}{%
    \setcounter{table}{0}
    \renewcommand{\thetable}{S\arabic{table}}%
    \setcounter{figure}{0}
    \renewcommand{\thefigure}{S\arabic{figure}}%
}

\def\changemargin#1#2{\list{}{\rightmargin#2\leftmargin#1}\item[]}
\let\endchangemargin=\endlist 

\title{Z chromosome dosage compensation in the northern shortfin squid, \textit{Illex illecebrosus}}
\author[1, 2]{Scott T. Small}
\author[1, 2]{Silas Tittes}
\author[3,4]{Thomas Desvignes}
\author[1,4]{John H. Postlethwait}
\author[1, 2]{Andrew D. Kern}
\affil[1]{\small{University of Oregon, Institute of Ecology and Evolution}}
\affil[2]{\small{University of Oregon, Department of Biology}}
\affil[3]{\small{Department of Biology, University of Alabama at Birmingham}}
\affil[4]{\small{University of Oregon, Institute of Neuroscience}}

\date{\small{\today{}}}

\begin{document}

\maketitle

\section*{Introduction}
Dosage compensation is the regulatory mechanism by which gene expression between
individuals with differing numbers of sex chromosomes is equalized. 
In organisms with heteromorphic sex chromosomes, such as XX/XY or ZZ/ZW systems,
gene content asymmetries can disrupt dosage-sensitive processes unless expression
levels are balanced. 
Multiple lineages have independently evolved mechanisms to address this imbalance. 
For example, in mammals, one X chromosome is transcriptionally silenced in females via the XIST long non-coding RNA \citep{lyon1961gene, brockdorff2015dosage}; 
in Drosophila, males upregulate their single X chromosome via the Male-Specific Lethal (MSL) complex \citep{lucchesi2015dosage}; 
and in Caenorhabditis elegans, hermaphrodites downregulate both X chromosomes by half \citep{meyer2005x}. 
These strategies ensure dosage parity with the heterogametic sex and preserve essential gene balance during development.

Despite its evolutionary utility, dosage compensation is not ubiquitous or mechanistically uniform. 
In birds (ZZ/ZW), females often exhibit lower expression of Z-linked genes 
compared to males, indicating incomplete compensation \citep{mank2009w, itoh2007dosage}.
Similar patterns are seen in Lepidoptera and some reptiles, 
where partial or gene-specific compensation suggests that selection for dosage balance 
is variable across lineages \citep{julien2012mechanisms, gu2017evolution}. 
The degree of compensation often correlates with the age and extent of degeneration
of the sex-limited chromosome. 
These observations imply that dosage compensation may evolve progressively 
and that partial regulation may suffice in lineages with relatively homomorphic or 
recently evolved sex chromosomes.

Cephalopods, long thought to lack genetic sex chromosomes, have recently emerged as a 
compelling group for studying the evolution of sex chromosomes. 
Classical karyotype analyses failed to reveal heteromorphic sex chromosomes in octopuses, squids, or cuttlefish (CITE), 
leading to hypotheses of environmental or polygenic sex determination. 
However, recent genomic work has overturned this view. 
\cite{coffing2025cephalopod} demonstrated that in octopuses, including \textit{Octopus bimaculoides}, females are hemizygous for a large chromosome (chr17), 
establishing a ZZ/ZO system shared across multiple coleoid lineages. 
This Z chromosome appears to be conserved across squids, cuttlefish, and octopuses, originating over 480 million years ago in a common ancestor. 
Complementary findings by \cite{torrado2025nautilus} in \textit{Nautilus pompilius}—which 
possesses an XX/XY system—highlight the evolutionary diversity of sex determination 
across cephalopods. 
These discoveries position coleoid cephalopods as one of the few known invertebrate 
groups with deeply conserved sex chromosomes.

The presence of a ZO/ZZ system in coleoids raises the question of how these animals
achieve dosage compensation between sexes, 
particularly in females that carry only one Z chromosome. 
Until recently, no studies had examined this issue in any cephalopod. 
\cite{papanicolaou2024z} recently addressed this gap 
by analyzing transcriptomes from \textit{Octopus vulgaris} and \textit{O. sinensis}, and found that Z-linked gene expression was partially compensated in females,
with male-to-female expression ratios averaging ~1.5:1. 
Moreover, they identified a male-specific long non-coding RNA (\textit{Zmast}) 
expressed from the Z chromosome, 
suggesting a potential regulatory mechanism for dosage modulation. 
While these results reveal the first evidence of dosage compensation in cephalopods, 
no comparable studies have been conducted in squids, 
leaving open fundamental questions about the prevalence, extent, 
and molecular underpinnings of compensation across the group. 
Notably, ommastrephid squids such as \illex lack prior genomic or transcriptomic characterization of their sex chromosomes.

In this study, we present a chromosome-scale genome assembly of the northern shortfin squid, 
\illex, and identify its Z chromosome via comparative genomic coverage and synteny analyses. 
We integrate sex-stratified transcriptomic data to assess expression differences between males and 
females for Z-linked genes, and test for evidence of dosage compensation. Finally, we ask if DNA 
methylation contributes to sex-specific regulation on the Z chromosome. T
his work represents the first comprehensive analysis of sex chromosome dosage compensation in any squid 
species and provides critical insight into how ancient sex chromosome systems are regulated in a major 
marine invertebrate lineage.

\section*{Results}


\begin{figure}[h]
    \centering
    \includegraphics[width=1\linewidth]{figures/cephalopod_tree_phylopics.pdf}
    \caption{Uncalibrated ultrametic phylogeny of Cephalopods with PhyloPic Silhouettes for each tip, shaded by clade.}
    \label{fig:phylo}
\end{figure}

\section*{Discussion}

\section*{Materials and Methods}

\section*{Acknowledgements}

\bibliography{refs}
\end{document}
